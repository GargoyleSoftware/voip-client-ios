\section{\-Basic registration}
\label{group__registration__tutorials}\index{\-Basic registration@{\-Basic registration}}
\-This program is a \-\_\-very\-\_\- simple usage example of liblinphone. \-Desmonstrating how to initiate a \-S\-I\-P registration from a sip uri identity passed from the command line. first argument must be like sip\-:{\tt jehan@sip.\-linphone.\-org} , second must be password . \par
 ex registration sip\-:{\tt jehan@sip.\-linphone.\-org} secret \par
\-Registration is cleared on \-S\-I\-G\-I\-N\-T \par
 
\begin{DoxyCodeInclude}

/*
linphone
Copyright (C) 2010  Belledonne Communications SARL 

This program is free software; you can redistribute it and/or
modify it under the terms of the GNU General Public License
as published by the Free Software Foundation; either version 2
of the License, or (at your option) any later version.

This program is distributed in the hope that it will be useful,
but WITHOUT ANY WARRANTY; without even the implied warranty of
MERCHANTABILITY or FITNESS FOR A PARTICULAR PURPOSE.  See the
GNU General Public License for more details.

You should have received a copy of the GNU General Public License
along with this program; if not, write to the Free Software
Foundation, Inc., 59 Temple Place - Suite 330, Boston, MA  02111-1307, USA.
*/

#ifdef IN_LINPHONE
#include "linphonecore.h"
#else
#include "linphone/linphonecore.h"
#endif

#include <signal.h>

static bool_t running=TRUE;

static void stop(int signum){
        running=FALSE;
}

static void registration_state_changed(struct _LinphoneCore *lc, 
      LinphoneProxyConfig *cfg, LinphoneRegistrationState cstate, const char *message
      ){
                printf("New registration state %s for user id [%s] at proxy
       [%s]\n"
                                ,linphone_registration_state_to_string(cstate)
                                ,linphone_proxy_config_get_identity(cfg)
                                ,linphone_proxy_config_get_addr(cfg));
}

LinphoneCore *lc;
int main(int argc, char *argv[]){
        LinphoneCoreVTable vtable={0};

        char* identity=NULL;
        char* password=NULL;

        /* takes   sip uri  identity from the command line arguments */
        if (argc>1){
                identity=argv[1];
        }

        /* takes   password from the command line arguments */
        if (argc>2){
                password=argv[2];
        }

        signal(SIGINT,stop);

#ifdef DEBUG
        linphone_core_enable_logs(NULL); /*enable liblinphone logs.*/
#endif
        /* 
         Fill the LinphoneCoreVTable with application callbacks.
         All are optional. Here we only use the registration_state_changed
       callbacks
         in order to get notifications about the progress of the registration.
         */
        vtable.registration_state_changed=registration_state_changed;

        /*
         Instanciate a LinphoneCore object given the LinphoneCoreVTable
        */
        lc=linphone_core_new(&vtable,NULL,NULL,NULL);

        LinphoneProxyConfig* proxy_cfg;
        /*create proxy config*/
        proxy_cfg = linphone_proxy_config_new();
        /*parse identity*/
        LinphoneAddress *from = linphone_address_new(identity);
        if (from==NULL){
                printf("%s not a valid sip uri, must be like
       sip:toto@sip.linphone.org \n",identity);
                goto end;
        }
                LinphoneAuthInfo *info;
                if (password!=NULL){
                        info=linphone_auth_info_new(
      linphone_address_get_username(from),NULL,password,NULL,NULL); /*create
       authentication structure from identity*/
                        linphone_core_add_auth_info(lc,info); /*add
       authentication info to LinphoneCore*/
                }

                // configure proxy entries
                linphone_proxy_config_set_identity(proxy_cfg,identity); /*set
       identity with user name and domain*/
                const char* server_addr = linphone_address_get_domain(from); /*
      extract domain address from identity*/
                linphone_proxy_config_set_server_addr(proxy_cfg,server_addr); 
      /* we assume domain = proxy server address*/
                linphone_proxy_config_enable_register(proxy_cfg,TRUE); /*
      activate registration for this proxy config*/
                linphone_address_destroy(from); /*release resource*/

                linphone_core_add_proxy_config(lc,proxy_cfg); /*add proxy
       config to linphone core*/
                linphone_core_set_default_proxy(lc,proxy_cfg); /*set to default
       proxy*/


        /* main loop for receiving notifications and doing background
       linphonecore work: */
        while(running){
                linphone_core_iterate(lc); /* first iterate initiates
       registration */
                ms_usleep(50000);
        }

        linphone_core_get_default_proxy(lc,&proxy_cfg); /* get default proxy
       config*/
        linphone_proxy_config_edit(proxy_cfg); /*start editing proxy
       configuration*/
        linphone_proxy_config_enable_register(proxy_cfg,FALSE); /*de-activate
       registration for this proxy config*/
        linphone_proxy_config_done(proxy_cfg); /*initiate REGISTER with expire
       = 0*/

        while(linphone_proxy_config_get_state(proxy_cfg) !=  
      LinphoneRegistrationCleared){
                linphone_core_iterate(lc); /*to make sure we receive call backs
       before shutting down*/
                ms_usleep(50000);
        }

end:
        printf("Shutting down...\n");
        linphone_core_destroy(lc);
        printf("Exited\n");
        return 0;
}

\end{DoxyCodeInclude}
 