\section{\-Basic call}
\label{group__basic__call__tutorials}\index{\-Basic call@{\-Basic call}}
\-This program is a \-\_\-very\-\_\- simple usage example of liblinphone. \-It just takes a sip-\/uri as first argument and attempts to call it


\begin{DoxyCodeInclude}

/*
linphone
Copyright (C) 2010  Belledonne Communications SARL 
 (simon.morlat@linphone.org)

This program is free software; you can redistribute it and/or
modify it under the terms of the GNU General Public License
as published by the Free Software Foundation; either version 2
of the License, or (at your option) any later version.

This program is distributed in the hope that it will be useful,
but WITHOUT ANY WARRANTY; without even the implied warranty of
MERCHANTABILITY or FITNESS FOR A PARTICULAR PURPOSE.  See the
GNU General Public License for more details.

You should have received a copy of the GNU General Public License
along with this program; if not, write to the Free Software
Foundation, Inc., 59 Temple Place - Suite 330, Boston, MA  02111-1307, USA.
*/

#ifdef IN_LINPHONE
#include "linphonecore.h"
#else
#include "linphone/linphonecore.h"
#endif

#include <signal.h>

static bool_t running=TRUE;

static void stop(int signum){
        running=FALSE;
}

/*
 * Call state notification callback
 */
static void call_state_changed(LinphoneCore *lc, LinphoneCall *call, 
      LinphoneCallState cstate, const char *msg){
        switch(cstate){
                case LinphoneCallOutgoingRinging:
                        printf("It is now ringing remotely !\n");
                break;
                case LinphoneCallOutgoingEarlyMedia:
                        printf("Receiving some early media\n");
                break;
                case LinphoneCallConnected:
                        printf("We are connected !\n");
                break;
                case LinphoneCallStreamsRunning:
                        printf("Media streams established !\n");
                break;
                case LinphoneCallEnd:
                        printf("Call is terminated.\n");
                break;
                case LinphoneCallError:
                        printf("Call failure !");
                break;
                default:
                        printf("Unhandled notification %i\n",cstate);
        }
}

int main(int argc, char *argv[]){
        LinphoneCoreVTable vtable={0};
        LinphoneCore *lc;
        LinphoneCall *call=NULL;
        const char *dest=NULL;

        /* take the destination sip uri from the command line arguments */
        if (argc>1){
                dest=argv[1];
        }

        signal(SIGINT,stop);

#ifdef DEBUG
        linphone_core_enable_logs(NULL); /*enable liblinphone logs.*/
#endif
        /* 
         Fill the LinphoneCoreVTable with application callbacks.
         All are optional. Here we only use the call_state_changed callbacks
         in order to get notifications about the progress of the call.
         */
        vtable.call_state_changed=call_state_changed;

        /*
         Instanciate a LinphoneCore object given the LinphoneCoreVTable
        */
        lc=linphone_core_new(&vtable,NULL,NULL,NULL);

        if (dest){
                /*
                 Place an outgoing call
                */
                call=linphone_core_invite(lc,dest);
                if (call==NULL){
                        printf("Could not place call to %s\n",dest);
                        goto end;
                }else printf("Call to %s is in progress...",dest);
                linphone_call_ref(call);
        }
        /* main loop for receiving notifications and doing background
       linphonecore work: */
        while(running){
                linphone_core_iterate(lc);
                ms_usleep(50000);
        }
        if (call && linphone_call_get_state(call)!=LinphoneCallEnd){
                /* terminate the call */
                printf("Terminating the call...\n");
                linphone_core_terminate_call(lc,call);
                /*at this stage we don't need the call object */
                linphone_call_unref(call);
        }

end:
        printf("Shutting down...\n");
        linphone_core_destroy(lc);
        printf("Exited\n");
        return 0;
}

\end{DoxyCodeInclude}
 