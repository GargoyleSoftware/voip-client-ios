\section{\-Chat room and messaging}
\label{group__chatroom__tuto}\index{\-Chat room and messaging@{\-Chat room and messaging}}
\-This program is a \-\_\-very\-\_\- simple usage example of liblinphone, desmonstrating how to send/receive \-S\-I\-P \-M\-E\-S\-S\-A\-G\-E from a sip uri identity passed from the command line. \par
\-Argument must be like sip\-:{\tt jehan@sip.\-linphone.\-org} . \par
 ex chatroom sip\-:{\tt jehan@sip.\-linphone.\-org} \par
 
\begin{DoxyCodeInclude}

/*
linphone
Copyright (C) 2010  Belledonne Communications SARL 

This program is free software; you can redistribute it and/or
modify it under the terms of the GNU General Public License
as published by the Free Software Foundation; either version 2
of the License, or (at your option) any later version.

This program is distributed in the hope that it will be useful,
but WITHOUT ANY WARRANTY; without even the implied warranty of
MERCHANTABILITY or FITNESS FOR A PARTICULAR PURPOSE.  See the
GNU General Public License for more details.

You should have received a copy of the GNU General Public License
along with this program; if not, write to the Free Software
Foundation, Inc., 59 Temple Place - Suite 330, Boston, MA  02111-1307, USA.
*/

#ifdef IN_LINPHONE
#include "linphonecore.h"
#else
#include "linphone/linphonecore.h"
#endif

#include <signal.h>

static bool_t running=TRUE;

static void stop(int signum){
        running=FALSE;
}
void text_received(LinphoneCore *lc, LinphoneChatRoom *room, const 
      LinphoneAddress *from, const char *message) {
        printf(" Message [%s] received from [%s] \n",message,
      linphone_address_as_string (from));
}


LinphoneCore *lc;
int main(int argc, char *argv[]){
        LinphoneCoreVTable vtable={0};

        char* dest_friend=NULL;


        /* takes   sip uri  identity from the command line arguments */
        if (argc>1){
                dest_friend=argv[1];
        }

        signal(SIGINT,stop);
//#define DEBUG
#ifdef DEBUG
        linphone_core_enable_logs(NULL); /*enable liblinphone logs.*/
#endif
        /* 
         Fill the LinphoneCoreVTable with application callbacks.
         All are optional. Here we only use the text_received callback
         in order to get notifications about incoming message.
         */
        vtable.text_received=text_received;

        /*
         Instantiate a LinphoneCore object given the LinphoneCoreVTable
        */
        lc=linphone_core_new(&vtable,NULL,NULL,NULL);


        /*Next step is to create a chat root*/
        LinphoneChatRoom* chat_room = linphone_core_create_chat_room(lc,
      dest_friend);

        linphone_chat_room_send_message(chat_room,"Hello world"); /*sending
       message*/

        /* main loop for receiving incoming messages and doing background
       linphone core work: */
        while(running){
                linphone_core_iterate(lc);
                ms_usleep(50000);
        }

        printf("Shutting down...\n");
        linphone_chat_room_destroy(chat_room);
        linphone_core_destroy(lc);
        printf("Exited\n");
        return 0;
}

\end{DoxyCodeInclude}
 